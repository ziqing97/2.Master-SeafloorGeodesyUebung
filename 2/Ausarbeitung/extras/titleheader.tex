\documentclass[%
paper=a4,      % alle weiteren Papierformat einstellbar
fontsize=11pt, % Schriftgr��e (12pt, 11pt (Standard))
BCOR1cm,       % Bindekorrektur, bspw. 1 cm
twoside,       % Doppelseiten
headsepline,   %
headings=openright, % Kapitel nur rechts beginnen
%biblography=totoc, % Literaturverzeichnis einf�gen bibtotocnumbered: nummeriert
parskip=half,  % Europ�ischer Satz mit Abstand zwischen Abs�tzen
chapterprefix, % Kapitel anschreiben als Kapitel
headsepline,   % Linie nach Kopfzeile
titlepage,     %
numbers=noenddot,
%draft	       % zeigt �berlange Zeilen an
]{scrreprt}

% anderer Satzspiegel f�r die Titelseite, deshalb in separaten Dokument
\usepackage[textwidth=15.5cm,textheight=26.7cm,top=1.5cm,left=3cm,
            ignoreall,noheadfoot,nomarginpar,footnotesep=0pt]{geometry} 

\usepackage[T1]{fontenc}
\usepackage[latin1]{inputenc}  % Zeichencodierung
\usepackage[ngerman, english]{babel} % Worttrennung nach neuer Rechtschreibung
\usepackage{ellipsis}       % Leerraum um Auslassungspunkte
\usepackage{fixltx2e}       % Fehlerkorrektur Zeichens�tze
\usepackage{xspace}         % f�ge evtl. notwendiges Leerzeichen hinzu (\xspace)

%\usepackage{mathptmx}           % Times + passende Mathefonts
\usepackage{mathpazo}           % Palatino + passende Mathefonts
\usepackage[scaled=.92]{helvet} % skalierte Helvetica als \sfdefault
\usepackage{courier}            % Courier als \ttdefault

\usepackage{graphicx}    % Einbindung von Grafiken
\graphicspath{{bilder/}} % Unterverzeichnis, in dem Grafiken abgelegt werden
\usepackage{listings}    % Listenausgabe externer Dateien
\usepackage{subfigure}

% Andere Schriftarten in Koma-Script
\setkomafont{sectioning}{\normalfont\bfseries}
\setkomafont{captionlabel}{\rmfamily\bfseries\small}
\setkomafont{caption}{\mdseries\itshape\small}
\setkomafont{pagehead}{\normalfont\itshape} % Kopfzeilenschrift
\setkomafont{descriptionlabel}{\normalfont\bfseries}

% weitere Einstellungen
\tolerance=200               % �bervolle Zeile vermeiden
\emergencystretch=3em

\clubpenalty=10000           % 'Schusterjungen' und 'Hurenkinder' vermeiden
\widowpenalty=10000 
\displaywidowpenalty=10000

\parindent 0pt               % Einzug zu Absatzbeginn festlegen

\setcapindent{1em}           % Zeilenumbruch bei Bildbeschreibungen.

\setcounter{secnumdepth}{3}  % Strukturiertiefe bis subsubsection{} m�glich
\setcounter{tocdepth}{3}     % Dargestellte Strukturiertiefe im Inhaltsverzeichnis

